\section{Exact Results}

% \subsection{Existance LE}
% \begin{frame}
% 	1 Frame\\
% 	Use Paper to show existance
% \end{frame}

\subsection{New Exact Solution for n=2}
\begin{frame}
	\frametitle{\insertsubsection}
	\begin{itemize}[<+->]
		\item Found new solution for $n=2$ by using simple power-series $\theta=\sum a_m\xi^m$
		\begin{equation}
			\frac{1}{\xi^2}\frac{\partial}{\partial\xi}\left(\xi^2\frac{\partial\theta}{\partial\xi}\right) +\theta^2 =0
		\end{equation}
		\item Apply Cauchy-Product formula and combine
		\begin{equation}
			(m+1)\sum\limits_{m=0}^\infty\left((m+2)a_{m+2}\xi^{m}+2a_{m+1}\xi^{m-1} + \sum\limits_{k=0}^m a_{m-k}a_k\xi^m\right) = 0
		\end{equation}
		\begin{theorem}
			The odd coefficients $a_{2m+1}$ vanish.
		\end{theorem}
		\item Proof by induction. Start at $\left.\partial\theta\right|_{\xi=0}=a_1=0$ is clear.
	\end{itemize}
\end{frame}


\begin{frame}
	\begin{itemize}[<+->]
		\item Obtain recursive Formula for coefficients
		\begin{equation}
			a_{2m+2}=b_{m+1}= -\frac{1}{(m+2)(m+3)}\sum\limits_{k=0}^mb_{m-k}b_k
		\end{equation}
		\begin{theorem}
			The series $\theta=\sum\limits_{m=0}^\infty b_{m}\xi^{2m}$ converges for $\xi<1$. % TODO was heißt absolute Konvergenz auf Englisch?
		\end{theorem}
		\begin{proof}
			We know that $a_0=1$. Use L'Hospital's rule for $\xi\rightarrow0$
			\begin{equation}
				0=\frac{2}{\xi}\frac{\partial\theta}{\partial\xi} + \frac{\partial^2\theta}{\partial\xi^2}+\theta^2\longrightarrow3\left.\frac{\partial^2\theta}{\partial\xi^2}\right|_{\xi=0}+\left.\theta\right|_{\xi=0}=0
			\end{equation}
			Thus $b_1=\theta_0/6$. By induction, we get $|b_{m+1}|\leq\frac{m\theta_0^2}{(m+2)(m+3)}$. Now use quotient criterion.
			% TODO find correct naming for Quotientenkriterium
		\end{proof}
	\end{itemize}
\end{frame}

\subsection{Hypothesis}
\begin{frame}
	\frametitle{\insertsubsection}
	\begin{figure}
		\scalebox{0.65}{\import{Part1-Main/Pictures/}{LE-ExactN2.pgf}}
		\caption{Exact solution $\theta$ and $(b_m)^{-1/(2m+2)}$ against $m$}
	\end{figure}
	\begin{itemize}%[<+->]
		\begin{theorem}
			The radius of convergence $R$ of the above calculated series is exactly the value at which $\theta(R)=0$.
		\end{theorem}
		\item Numerics: Calculation not optimized, reaches floating point limit
	\end{itemize}
\end{frame}


\subsection{Limiting Case TOV}
\begin{frame}[allowframebreaks]
% 	1 Frame\\
% 	Use Assumption that there exists a cont. solution of the TOV equation
	\frametitle{\insertsubsection}
	\begin{itemize}[<+->]
		\item Suppose the TOV equation has a solution that continuously depends on its parameters for $r\in[0,l)$ where $l$ may be $\infty$.
		\begin{theorem}
			Let $p_A$ be a solution of the TOV equation with $\rho=Ap^{1/\gamma}$. Then 
			\begin{equation}
				\lim_{A\rightarrow0}p_A=\frac{p_0}{2\pi p_0r^2+1}
			\end{equation}
		\end{theorem}
		\begin{proof}
			With $\partial m/\partial r = 4\pi Ap^{1/\gamma}r^2$, define $v:=m/A$
			\begin{equation}
				\frac{\partial p}{\partial r} = -\frac{p^{1/\gamma}}{r^2}\left(A+p^{1-1/\gamma}\right)\left(4\pi r^3p+vA\right)\left(1-\frac{2vA}{r}\right)^{-1}
			\end{equation}
		\end{proof}
	\end{itemize}
\end{frame}

\begin{frame}
	\begin{itemize}[<+->]
		\begin{proof}
			Then for $A=0$, we have
			\begin{equation}
				\frac{\partial p}{\partial r} = -4\pi rp^2
			\end{equation}
			The solution to this differential equation is $p=\frac{p_0}{2\pi p_0r^2+1}$.
		\end{proof}
		\item Problem: Assumption needs that TOV equation is solvable
		\item No experience with 2D singular ODEs.
		\item Easy transformation tricks for 1D singular ODEs not working.
	\end{itemize}
\end{frame}


% \subsection{}
