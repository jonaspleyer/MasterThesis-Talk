\section{Numerical Solutions}

\subsection{Comparison of TOV and LE results}

% \begin{frame}
% 	\frametitle{\insertsubsection}
% 	\begin{table}[H]
% 		\renewcommand{\arraystretch}{1.2}
% 		\centering
% 		\begin{tabular}{@{}llcll@{}}
% 			\toprule
% 			\multicolumn{2}{c}{\textbf{TOV}} & \phantom{b} &\multicolumn{2}{c}{\textbf{LE}}\\
% 			\cmidrule{1-2} \cmidrule{4-5}
% 			EOS & $\rho=Ap^{1/\gamma}$ && EOS & $p=K\rho^{\gamma}$\\
% 			$A$ & $2$ & & \\
% 			$\gamma=1+\frac{1}{n}$ & $4/3$ && $n=1/(\gamma-1)$ & $3$\\
% 			$p_0$ & $0.5$ && $\theta_0$ & $1$\\
% 			$m_0$ & $0$ && $d\theta_0$ & $0$\\
% 			$dr$ & $0.01$ && $d\xi=dr/\beta$ & $\frac{0.01}{0.3355}\approx0.0298$\\
% 			\cmidrule{1-2} \cmidrule{4-5}
% 			$\rho_0=Ap_0^{1/\gamma}$ & $2(2)^{\frac{4}{3}}\approx1.1892$ && $\lambda=\rho_0$ & $2(2)^{\frac{4}{3}}\approx1.1892$\\
% 			&&& $K=A^{-1/\gamma}$ & $2^{-3/4}\approx0.5946$\\
% 			&&& $\beta^2=\frac{(n+1)K\lambda^{1/n-1}}{4\pi}$ & $\approx0.1125$\\
% 			\bottomrule
% 		\end{tabular}
% 	\end{table}%
% \end{frame}



\begin{frame}
	\begin{figure}
		\centering
		\scalebox{0.6}{\import{Part1-Main/Pictures/}{TOV-LE-Combi.pgf}}
		\caption{Comparison of TOV and LE results}
	\end{figure}
\end{frame}

\subsection{Verifying LE Results}
\begin{frame}
	\frametitle{\insertsubsection}
	\begin{figure}
		\centering
		\scalebox{0.6}{\import{Part1-Main/Pictures/}{LE-ValidateSols.pgf}}
		\caption{Validation of numerically calculated Lane-Emden results}
	\end{figure}
\end{frame}

% \begin{frame}
% 	\frametitle{Comparing Partial TOV and LE results}
% 	\begin{itemize}
% 		\item Next, reduce terms present in TOV equation
% 		\begin{equation}
% 			\frac{\partial p}{\partial r} = -\underbrace{\underbrace{\underbrace{\underbrace{\frac{Gm\rho}{r^2}}_{i=0}\hspace{0.1cm}\left(1+\frac{p}{\rho c^2}\right)}_{i=1}\hspace{0.1cm}\left(\frac{4\pi r^3p}{mc^2}+1\right)}_{i=2}\hspace{0.1cm}\left(1-\frac{2Gm}{rc^2}\right)^{-1}}_{i=3}
% 		\end{equation}
% 		\item We expect $i=0$ to be the LE equation as derived earlier
% 	\end{itemize}
% \end{frame}
% 
% 
% \begin{frame}
% 	\begin{figure}
% 		\centering
% 		\scalebox{0.6}{\import{Part1-Main/Pictures/}{TOV-Terms.pgf}}
% 	\end{figure}
% \end{frame}


\subsection{Zero Values of TOV and LE equation}
\begin{frame}
	\frametitle{\insertsubsection}
% 	Show all the nice plots with the $n$ vs $\xi_0$ for different parameters plotted
	\begin{figure}
		\centering
		\scalebox{0.65}{\import{Part1-Main/Pictures/}{TOV-Exponents-LESubs-InitialVals-Database-PlotResults.pgf}}
		\caption{TOV and LE results for varying $A$ parameter of $\rho=Ap^{n/(n+1)}$}
	\end{figure}
\end{frame}

\begin{frame}
	\begin{figure}
		\centering
		\scalebox{0.65}{\import{Part1-Main/Pictures/}{TOV-Exponents-LESubs-InitialVals-Database-PlotResults-1.pgf}}
		\caption{TOV and LE solutions for varying parameter $p_0$.}
	\end{figure}
	Intersection: $r=\beta\xi$ with $4\pi\beta A^{n/(n+1)}=(n+1)\lambda^{1-1/n}$ is independent of $\lambda=\rho(p_0)$.
\end{frame}

\subsection{TOV Hypothesis}
\begin{frame}
	\frametitle{\insertsubsection}
	\begin{hypothesis}
		Given the TOV differential equation with $\rho=Ap^{\frac{n}{n+1}}$ and $p_0,A>0$
		\begin{alignat*}{3}
			\frac{\partial m}{\partial r} &= &&4\pi\rho r^2\\
			\frac{\partial p}{\partial r} &= -&&\frac{m\rho}{r^2}\left(1+\frac{p}{\rho}\right)\left(\frac{4\pi r^3p}{m}+1\right)\left(1-\frac{2m}{r}\right)^{-1}
		\end{alignat*}
		\begin{enuemrate}
			\item There exists a solution for some $n_0\leq0$
			\item such that all solutions with same parameters $A,p_0$ and smaller exponent $n\leq n_0$ have a zerovalue for some $\xi>0$.
		\end{enumerate}
% 
% 		
% 		
% 		To each combination $p_0,A>0$ there exists a $n_0\geq0$ such that all solutions of the TOV differential equation
% 		\begin{alignat*}{3}
% 			\frac{\partial m}{\partial r} &= &&4\pi\rho r^2\\
% 			\frac{\partial p}{\partial r} &= -&&\frac{m\rho}{r^2}\left(1+\frac{p}{\rho}\right)\left(\frac{4\pi r^3p}{m}+1\right)\left(1-\frac{2m}{r}\right)^{-1}
% 		\end{alignat*}
% 		with $\rho=Ap^{\frac{n}{n+1}}$ and a smaller $n\leq n_0$ has $p(\xi_0)=0$ for some $\xi_0\in\R_{>0}$.
	\end{hypothesis}
\end{frame}


% \subsection{Summary of Numerical Results}
% \begin{frame}
% 	\frametitle{\insertsubsection}
% 	\begin{itemize}
% 		\item Numerical and Exact results agree well where exact solutions are known
% 		
% 	\end{itemize}
% \end{frame}
