\section{General Relativity}

\subsection{Concepts}
\begin{frame}
	\frametitle{\insertsubsection}
	TODO include nice picture %TODO
	\begin{itemize}%[<+->]
		\item General relativity (GR) models large scale of measurable universe
		\item Lorentzian Geometry (with indefinitive metric $g_{\mu\nu}$)
		\item Einstein equations relate curvature with mass and energy
		\[G_{\mu\nu} + \Lambda g_{\mu\nu}=R_{\mu\nu}+\left(\frac{1}{2}R+\Lambda\right)g_{\mu\nu}=8\pi T_{\mu\nu}\]
		\item Heavy objects create curvature in space
	\end{itemize}
\end{frame}


\subsection{Deriving the TOV equation}
\begin{frame}
	\frametitle{\insertsubsection}
	\begin{itemize}%[<+->]
		\item TOV equation was first derived 
		independently by Tolman \cite{tolmanStaticSolutionsEinstein1939} and 
		Oppenheimer with Volkoff \cite{oppenheimerMassiveNeutronCores1939}.
		\item Spherically symmetric (Lorentz) metric 
		\begin{equation}
			g=-\e^\nu dt^2+\e^\lambda dr^2+r^2\left(d\theta^2+\sin^2\theta d\phi^2\right)
		\end{equation}
		\item Energy-Momentum Tensor of perfect fluid
		\begin{equation}
			T_{\mu\nu}=\text{diag}(-\rho,p,p,p)
		\end{equation}
		\item Solve Einstein Equations (without cosm. constant)
		\begin{equation}
			G_{\mu\nu}=R_{\mu\nu}+\frac{1}{2}Rg_{\mu\nu}=8\pi T_{\mu\nu}
		\end{equation}
	\end{itemize}
\end{frame}

\begin{frame}
	\begin{itemize}%[<+->]
		\item Obtain 3 distinct differential equations ($R_{33}=R_{22}$)
		\begin{align}
			-8\pi T_0^0 = 8\pi\rho &= \frac{\lambda'\e^{-\lambda}}{r}+\frac{1-\e^{-\lambda}}{r^2}\label{3-Mass-EinstEqu-1}\\
			8\pi T_1^1 = 8\pi p &= \nu'\frac{\e^{-\lambda}}{r} - \frac{1-\e^{-\lambda}}{r^2}\label{3-Mass-EinstEqu-2}\\
			8\pi T_2^2 = 8\pi p &= \frac{\e^{-\lambda}}{2}\left[\nu''+\left(\frac{\nu'}{2} + \frac{1}{r}\right)\left(\nu'-\lambda'\right) \right]\label{3-Mass-EinstEqu-3}
		\end{align}
		\item Use equation \ref{3-Mass-EinstEqu-1} and identify Mass $m(r)$
		\begin{equation}
			\e^{-\lambda}=1-\frac{2}{r}\int\limits_0^r 4\pi\rho(r')r'^2dr' =: 1-\frac{2m(r)}{r} 
			\label{3-Mass-MassDefinition}
		\end{equation}
		% TODO better formulation!
		\item Divergence of Energy-Momentum Tensor is $\nabla_{\mu}T^{\mu\nu}=0$. Then obtain
		\begin{equation}
			\frac{\partial p}{\partial r} = -\frac{p+\rho}{2}\frac{\partial \nu}{\partial r}
			\label{3-Mass-EinstEquation2-Divergence}
		\end{equation}
	\end{itemize}
\end{frame}

\begin{frame}
	\begin{itemize}%[<+->]
		\item Use equations \ref{3-Mass-EinstEqu-2}, \ref{3-Mass-MassDefinition} and \ref{3-Mass-EinstEquation2-Divergence} to get
		\begin{alignat}{3}
			\frac{\partial m}{\partial r} &= &&4\pi\rho r^2\label{3-Mass-TOV-Equation1}\\
			\frac{\partial p}{\partial r} &= -&&\frac{Gm\rho}{r^2}\left(1+\frac{p}{\rho c^2}\right)\left(\frac{4\pi r^3p}{mc^2}+1\right)\left(1-\frac{2Gm}{rc^2}\right)^{-1}\label{3-Mass-TOV-Equation2}
		\end{alignat}
		\item Plugged in gravitational constant, speed of light $G=c=1$
		\item Equation \ref{3-Mass-TOV-Equation1} from Mass-Definition
		\item Ordinary differential equation
		\item Singular at $r=0$
		\item Needs equation of state (EOS) $f(\rho,p,r)=0$ to be solvable
	\end{itemize}
\end{frame}

\subsection{Newtonian Limit}
\begin{frame}
	\frametitle{\insertsubsection}
	\begin{itemize}%[<+->]
		\item Non-relativistic Limit of 2nd TOV equation \ref{3-Mass-TOV-Equation2} is
		\begin{equation}\frac{
			\partial p}{\partial r}=-\frac{Gm\rho}{r^2}
		\end{equation}
		\item Polytropic EOS $p=K\rho^{1+1/n}$
		\item Transformation $\rho=\lambda\theta^n$ and $\xi=r/\beta$ where
		\begin{equation}
			4\pi\beta^2=(n+1)K\lambda^{1-1/n}
		\end{equation}
		\item Obtain Lane-Emden equation \cite{laneTheoreticalTemperatureSun1870} \cite{emdenGaskugeln1907}
		\begin{equation}
			\frac{1}{\xi^2}\frac{\partial}{\partial\xi}\left(\xi^2\frac{\partial\theta}{\partial\xi}\right)+\theta^n=0
		\end{equation}
		\item Some exact solutions are known \cite{chandrasekharChandrasekharAnIntroductionStudy1958}
	\end{itemize}
\end{frame}

\begin{frame}
% 	\renewcommand{\arraystretch}{1.2}
	\begin{figure}
		\centering
		\import{Part1-Introduction/pictures/}{LE-SingleSolve.pgf}
		\begin{tabular}[b]{@{}lcccrl@{}}
			\toprule
			$n=0$ && $\displaystyle 1-1/6\xi^2$ && $\xi_0=$&$\sqrt{6}$\\[1ex]
			$n=1$ && $\displaystyle \sin(\xi)/\xi$ && $\xi_0=$&$\pi$\\[1ex]
			$n=5$ && $\displaystyle \left(1+1/3\xi^2\right)^{-1/2}$ && $\xi_0=$&$\infty$\\[1ex]
			\bottomrule
		\end{tabular}
	\end{figure}
\end{frame}

% \subsection{Upper Mass Bounds}
% \begin{frame}
% 	\frametitle{\insertsubsection}
% 	2 Frames\\
% 	Show that $M/R<4/9$ is an upper bound.
% \end{frame}
