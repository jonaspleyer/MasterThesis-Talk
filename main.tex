\documentclass{beamer}
\usepackage[utf8]{inputenc}

\usepackage {amssymb}
\usepackage {amsthm}
\usepackage [english]{babel}
\usepackage [sorting=none, style=alphabetic, citestyle=alphabetic, url=false, backend=biber]
			{biblatex}
\usepackage {booktabs}
\usepackage [skip=0pt, belowskip=0pt]{caption}
\usepackage {csquotes}
\usepackage [T1]{fontenc}
\usepackage {graphicx}
\usepackage {hyperref}
\usepackage {import}
\usepackage [utf8]{inputenc}
\usepackage {lmodern}
\usepackage {mathtools}
\usepackage [protrusion=true,expansion=true,kerning]{microtype}
\usepackage {nameref}
\usepackage {pgf}
\usepackage {tikz}
\usepackage {xcolor}

% pgfplots and settings
\usepackage{pgfplots}
\pgfplotsset{compat=newest}
\usepgfplotslibrary{groupplots}
\usepgfplotslibrary{dateplot}

% Theme of the presentation
\usetheme{CambridgeUS}

\usecolortheme{dolphin}

% Declare this - symbol to import pgf files correctly
\DeclareUnicodeCharacter{2212}{-}

% Specify TopRule and Bottomrule Thickness for booktabs usepackage
\renewcommand{\heavyrulewidth}{2pt}

% Define colors of the university Freiburg hy their html value
\definecolor{Uni-Fr-Red}{HTML}{C41330}
\definecolor{Uni-Fr-Dark-Blue}{HTML}{00376d}
\definecolor{Uni-Fr-Mid-Blue}{HTML}{004799}
\definecolor{Uni-Fr-Light-Blue}{HTML}{5480c1}
% Apply those colors to the presentation
\setbeamercolor{normal text}{fg=black,bg=Uni-Fr-Light-Blue!30}
\setbeamercolor{alerted text}{fg=black}
\setbeamercolor{example text}{fg=black}
\setbeamercolor{background canvas}{fg=Uni-Fr-Light-Blue, bg=white}
\setbeamercolor{palette primary}{fg=white, bg=Uni-Fr-Light-Blue}
\setbeamercolor{palette secondary}{fg=white, bg=Uni-Fr-Mid-Blue}
\setbeamercolor{palette tertiary}{fg=white, bg=Uni-Fr-Dark-Blue}

% choose a very! good title
\title[TOV Equation and Star Masses]{The TOV Equation and the Mass of Stars}
% \subtitle{... at least not too bad.}

% For short you could for example use the last name only, it's optional as is
% the short title
\author[Jonas Pleyer]{Jonas Pleyer}

% Can be set, for students usually not required
%\institute{}

\date{\today}

\setbeamertemplate{section in toc}[sections numbered]
\setbeamertemplate{subsection in toc}[subsections numbered]

% Removes the navigation symbols
\beamertemplatenavigationsymbolsempty{}

% Add the bibliography for the project
\addbibresource{98-literature.bib}

% Add Large and small letters to be breakable in urls and DOIs in biblatex
\setcounter{biburllcpenalty}{9999}% Kleinbuchstaben
\setcounter{biburlucpenalty}{9999}% Großbuchstaben

% Sets linebreaking in names - Higher value = Dislikes such breaks
\setcounter{highnamepenalty}{9999}
\setcounter{lownamepenalty}{9999}

% Symbols for Number sets and other stuff
\newcommand{\Z}{\mathcal{Z}}
\newcommand{\F}{\mathcal{F}}
\newcommand{\Ent}{\mathcal{S}}
\newcommand{\U}{\mathcal{U}}
\newcommand{\R}{\mathbb{R}}
\newcommand{\e}{\mathit{e}}
\newcommand{\RP}{\mathbb{R}\text{P}}

% If you want that at the beginning of each section the table of contents
% is shown, I don't like it for short presentations
\AtBeginSection[]
{
  \begin{frame}
    \frametitle{Table of Contents}
    \tableofcontents[currentsection, hideothersubsections]
  \end{frame}
}


\begin{document}

\begin{frame}
  \titlepage
\end{frame}

\begin{frame}
	\frametitle{Motivation}
	\begin{itemize}[<+->]
		\item TOV equation: differential equation, describes stars in GR
		\item Needs additional information to be solvable
		\item Thermodynamics yields this information
		\item Questions to answer:
		\begin{itemize}
			\item Well defined radius?
			\item Mass limits?
		\end{itemize}
		\item Mathematically interesting: show that a solution of the differential equation has a zerovalue without knowing the solution
	\end{itemize}
\end{frame}


\begin{frame}
    \frametitle{Table of Contents}
	\tableofcontents[hideallsubsections,subsubsectionstyle=hide]
\end{frame}

\section{General Relativity}

\subsection{Concepts}
\begin{frame}
	\frametitle{\insertsubsection}
	TODO include nice picture %TODO
	\begin{itemize}[<+->]
		\item General relativity (GR) models large scale structure of measurable universe
		\item Lorentzian Geometry (with indefinitive metric $g_{\mu\nu}$)
		\item Einstein equations relate curvature with mass and energy (geometrized units $G=c=1$)
		\[G_{\mu\nu} + \Lambda g_{\mu\nu}=R_{\mu\nu}+\left(\frac{1}{2}R+\Lambda\right)g_{\mu\nu}=8\pi T_{\mu\nu}\]
		\item Heavy objects create curvature in space
	\end{itemize}
\end{frame}


\subsection{Deriving the TOV equation}
\begin{frame}
	\frametitle{\insertsubsection}
	\begin{itemize}[<+->]
		\item TOV equation was first derived 
		independently by Tolman \cite{tolmanStaticSolutionsEinstein1939} and 
		Oppenheimer with Volkoff \cite{oppenheimerMassiveNeutronCores1939}.
		\item Spherically symmetric (Lorentz) metric 
		\begin{equation}
			g=-\e^\nu dt^2+\e^\lambda dr^2+r^2\left(d\theta^2+\sin^2\theta d\phi^2\right)
		\end{equation}
		\item Energy-Momentum Tensor of perfect fluid
		\begin{equation}
			T_{\mu\nu}=\text{diag}(-\rho,p,p,p)
		\end{equation}
		\item Solve Einstein Equations (without cosm. constant)
		\begin{equation}
			G_{\mu\nu}=R_{\mu\nu}+\frac{1}{2}Rg_{\mu\nu}=8\pi T_{\mu\nu}
		\end{equation}
	\end{itemize}
\end{frame}

\begin{frame}
	\begin{itemize}[<+->]
		\item Obtain 3 distinct differential equations ($R_{33}=R_{22}$)
		\begin{align}
			-8\pi T_0^0 = 8\pi\rho &= \frac{\lambda'\e^{-\lambda}}{r}+\frac{1-\e^{-\lambda}}{r^2}\label{3-Mass-EinstEqu-1}\\
			8\pi T_1^1 = 8\pi p &= \nu'\frac{\e^{-\lambda}}{r} - \frac{1-\e^{-\lambda}}{r^2}\label{3-Mass-EinstEqu-2}\\
			8\pi T_2^2 = 8\pi p &= \frac{\e^{-\lambda}}{2}\left[\nu''+\left(\frac{\nu'}{2} + \frac{1}{r}\right)\left(\nu'-\lambda'\right) \right]\label{3-Mass-EinstEqu-3}
		\end{align}
		\item Use equation \ref{3-Mass-EinstEqu-1} and identify Mass $m(r)$
		\begin{equation}
			\e^{-\lambda}=1-\frac{2}{r}\int\limits_0^r 4\pi\rho(r')r'^2dr' =: 1-\frac{2m(r)}{r} 
			\label{3-Mass-MassDefinition}
		\end{equation}
		% TODO better formulation!
		\item Divergence of Energy-Momentum Tensor is $\nabla_{\mu}T^{\mu\nu}=0$. Then obtain
		\begin{equation}
			\frac{\partial p}{\partial r} = -\frac{p+\rho}{2}\frac{\partial \nu}{\partial r}
			\label{3-Mass-EinstEquation2-Divergence}
		\end{equation}
	\end{itemize}
\end{frame}

\begin{frame}
	\begin{itemize}[<+->]
		\item Use equations \ref{3-Mass-EinstEqu-2}, \ref{3-Mass-MassDefinition} and \ref{3-Mass-EinstEquation2-Divergence} to get
		\begin{alignat}{3}
			\frac{\partial m}{\partial r} &= &&4\pi\rho r^2\label{3-Mass-TOV-Equation1}\\
			\frac{\partial p}{\partial r} &= -&&\frac{Gm\rho}{r^2}\left(1+\frac{p}{\rho c^2}\right)\left(\frac{4\pi r^3p}{mc^2}+1\right)\left(1-\frac{2Gm}{rc^2}\right)^{-1}\label{3-Mass-TOV-Equation2}
		\end{alignat}
		\item Plugged in gravitational constant, speed of light $G=c=1$
		\item Equation \ref{3-Mass-TOV-Equation1} from Mass-Definition
		\item Ordinary differential equation
		\item Singular at $r=0$
		\item Needs equation of state (EOS) $f(\rho,p,r)=0$ to be numerically solvable
	\end{itemize}
\end{frame}

\subsection{Newtonian Limit}
\begin{frame}
	\frametitle{\insertsubsection}
	\begin{itemize}[<+->]
		\item Non-relativistic Limit of 2nd TOV equation \ref{3-Mass-TOV-Equation2} is
		\begin{equation}\frac{
			\partial p}{\partial r}=-\frac{Gm\rho}{r^2}
		\end{equation}
		\item Ansatz: Polytropic EOS $p=K\rho^{1+1/n}$
		\item Transformation $\rho=\lambda\theta^n$ and $\xi=r/\beta$ where
		\begin{equation}
			4\pi\beta^2=(n+1)K\lambda^{1-1/n}
		\end{equation}
		\item Obtain Lane-Emden equation \cite{laneTheoreticalTemperatureSun1870} \cite{emdenGaskugeln1907}
		\begin{equation}
			\frac{1}{\xi^2}\frac{\partial}{\partial\xi}\left(\xi^2\frac{\partial\theta}{\partial\xi}\right)+\theta^n=0
		\end{equation}
		\item Some exact solutions are known \cite{chandrasekharChandrasekharAnIntroductionStudy1958}
	\end{itemize}
\end{frame}

\begin{frame}
% 	\renewcommand{\arraystretch}{1.2}
	\begin{figure}
		\centering
		\scalebox{0.8}{\import{Part0-Introduction/pictures/}{LE-SingleSolve.pgf}}
		\begin{tabular}[b]{@{}lcccrl@{}}
			\toprule
			$n=0$ && $\displaystyle 1-1/6\xi^2$ && $\xi_0=$&$\sqrt{6}$\\[1ex]
			$n=1$ && $\displaystyle \sin(\xi)/\xi$ && $\xi_0=$&$\pi$\\[1ex]
			$n=5$ && $\displaystyle \left(1+1/3\xi^2\right)^{-1/2}$ && $\xi_0=$&$\infty$\\[1ex]
			\multicolumn{6}{c}{$\rho=\lambda\theta^n$ and $p=K\rho^{n+1/n}$}\\[1ex]
			\bottomrule
		\end{tabular}
	\end{figure}
\end{frame}

% \subsection{Upper Mass Bounds}
% \begin{frame}
% 	\frametitle{\insertsubsection}
% 	2 Frames\\
% 	Show that $M/R<4/9$ is an upper bound.
% \end{frame}

\section{Thermodynamics - Calculating an EoS}

\subsection{Short Summary of main Principles}
\begin{frame}
	\frametitle{\insertsubsection}
	\begin{itemize}[<+->]
		\item Statistical theory of manyparticle systems
		\item Describe macroscopic Phenomena by microscopic principles
		\item Partition function contains all information about $N$ particles with position $x_i\in M$ and momentum $p_i\in T_{x_i}M$ with $V=\text{vol}(M)$
		\begin{equation}
			\mathcal{Z}(T,V,N) = \int\limits_{TM^N}\exp\left(-\frac{H(x_1,\dots,p_N)}{k_BT}\right)\frac{dx_1\dots dp_N}{N!h^{3N}}
		\end{equation}
		\item Calculate equation of state via internal Energy $\mathcal{U}$ and
		\begin{equation}
			p = k_BT\frac{1}{\mathcal{Z}}\frac{\partial\mathcal{Z}}{\partial V} \hspace{1cm} \rho = \frac{\mathcal{U}}{V} = \frac{k_BT^2}{V}\frac{\partial\mathcal{Z}}{\partial T}
			\label{1-Thermo-Pres-Dens}
		\end{equation}
	\end{itemize}
\end{frame}

\begin{frame}
	\begin{itemize}[<+->]
		\item Partition Function for $H=\sqrt{m^2+p^2}$
		\begin{equation}
			\mathcal{Z} = \frac{1}{N!}\left(8\pi V\left(\frac{k_BT}{hc}\right)^3\frac{\alpha^2 K_2(\alpha)}{2}\right)^N
			\label{1-Thermo-Part-Func-Final}
		\end{equation}
		\item $K_2$ is modified Bessel function of 2nd kind
		\item Substitution $\alpha=mc^2/k_BT$
		\item With equations \ref{1-Thermo-Pres-Dens} and \ref{1-Thermo-Part-Func-Final} obtain
		\begin{equation}
			p = \frac{Nk_BT}{V} = CNmc^2\frac{1}{K_2(\alpha)\alpha^2}\exp\left(-\alpha\frac{K_1(\alpha)+K_3(\alpha)}{2K_2(\alpha)}\right)
			\label{1-Thermo-Pres-Alpha-Rel}
		\end{equation}
		\begin{theorem}
			The function $p(\alpha)$ above is a bijection.
		\end{theorem}
		\item Sketch of proof: take differential, use properties/representations of $K_\nu$
		\item Use inverse and previous equations \ref{1-Thermo-Pres-Dens}, \ref{1-Thermo-Part-Func-Final} and \ref{1-Thermo-Pres-Alpha-Rel} to obtain EOS
		\begin{equation}
			\rho = \frac{\U}{V} = p\left(1+\alpha(p)\frac{K_1(\alpha(p))+K_3(\alpha(p))}{2K_2(\alpha(p))}\right)
		\end{equation}
	\end{itemize}
\end{frame}


% \subsection{Non- and Ultra-Relativistic Case}
% \begin{frame}
% 	1 Frame\\
% 	Calculate the EoS for both cases
% \end{frame}

% \subsection{Fully Special Relativistic EoS}
% \begin{frame}
% 	\frametitle{\insertsubsection}
% 	\begin{itemize}%[<+->]
% 		\item Fully special relativistic Hamiltonian (single particle)
% 		\begin{equation}
% 			H = mc^2\sqrt{1+\frac{p^2}{m^2c^2}}
% 		\end{equation}
% 		\item Hamiltonian is independent of space-coordinate
% 		\item Plug into partition function (spherical coordinates)
% 		\begin{equation}
% 			\mathcal{Z} = \frac{V^N}{N!h^{3N}}\left[\int4\pi p^2\exp\left(-\frac{mc^2}{k_BT}\sqrt{1+\frac{p^2}{m^2c^2}}\right)dp\right]^N
% 		\end{equation}
% 		\item Lots of calculation later ...
% 		\begin{equation}
% 			\mathcal{Z} = \frac{1}{N!}\left(8\pi V\left(\frac{k_BT}{hc}\right)^32\alpha K_2(\alpha)\right)^N
% 		\end{equation}
% 		with $\alpha=mc^2/k_BT$ and $K_2$ modified Bessel function of the second kind
% 	\end{itemize}
% \end{frame}
% 
% 
% \begin{frame}
% 	\begin{itemize}%[<+->]
% 		\item Internal Energy
% 		\begin{equation}
% 			\mathcal{U} = 3Nk_BT-Nk_BT\left(\alpha\frac{\partial K_2(\alpha)}{K_2(\alpha)} + 2\right)
% 		\end{equation}
% 		\item Ultra relativistic limit is obtained when $m\rightarrow0$ or $\alpha\rightarrow0$
% 		\begin{equation}
% 			\mathcal{U}_{UR} = 3Nk_BT \hspace{1cm} \mathcal{Z}_{UR} = \frac{1}{N!}\left(8\pi\left(\frac{k_BT}{hc}\right)^3\right)
% 		\end{equation}
% 		\item Additionally assume adiabatic condition $\delta Q=0\Rightarrow\delta W=-pdV=dU$. This gives us
% 		\begin{align}
% 			-pdV &= C_VdT\\
% 			-\frac{Nk_BT}{V}dV &= N_kB\left[1+\alpha^2\left(\left(\frac{\partial_\alpha K_2(\alpha)}{K_2(\alpha)}\right)^2-\frac{\partial^2K_2(\alpha)}{K_2(\alpha)}\right)\right]dT
% 		\end{align}
% 	\end{itemize}
% \end{frame}
% 
% 
% \begin{frame}
% 	\begin{itemize}%[<+->]
% 		\item 
% 	\end{itemize}
% \end{frame}
% 

\section{Numerical Solutions}

\subsection{Comparison of TOV and LE results}

% \begin{frame}
% 	\frametitle{\insertsubsection}
% 	\begin{table}[H]
% 		\renewcommand{\arraystretch}{1.2}
% 		\centering
% 		\begin{tabular}{@{}llcll@{}}
% 			\toprule
% 			\multicolumn{2}{c}{\textbf{TOV}} & \phantom{b} &\multicolumn{2}{c}{\textbf{LE}}\\
% 			\cmidrule{1-2} \cmidrule{4-5}
% 			EOS & $\rho=Ap^{1/\gamma}$ && EOS & $p=K\rho^{\gamma}$\\
% 			$A$ & $2$ & & \\
% 			$\gamma=1+\frac{1}{n}$ & $4/3$ && $n=1/(\gamma-1)$ & $3$\\
% 			$p_0$ & $0.5$ && $\theta_0$ & $1$\\
% 			$m_0$ & $0$ && $d\theta_0$ & $0$\\
% 			$dr$ & $0.01$ && $d\xi=dr/\beta$ & $\frac{0.01}{0.3355}\approx0.0298$\\
% 			\cmidrule{1-2} \cmidrule{4-5}
% 			$\rho_0=Ap_0^{1/\gamma}$ & $2(2)^{\frac{4}{3}}\approx1.1892$ && $\lambda=\rho_0$ & $2(2)^{\frac{4}{3}}\approx1.1892$\\
% 			&&& $K=A^{-1/\gamma}$ & $2^{-3/4}\approx0.5946$\\
% 			&&& $\beta^2=\frac{(n+1)K\lambda^{1/n-1}}{4\pi}$ & $\approx0.1125$\\
% 			\bottomrule
% 		\end{tabular}
% 	\end{table}%
% \end{frame}



\begin{frame}
	\begin{figure}
		\centering
		\scalebox{0.6}{\import{Part1-Main/Pictures/}{TOV-LE-Combi.pgf}}
		\caption{Comparison of TOV and LE results}
	\end{figure}
\end{frame}

\subsection{Verifying LE Results}
\begin{frame}
	\frametitle{\insertsubsection}
	\begin{figure}
		\centering
		\scalebox{0.6}{\import{Part1-Main/Pictures/}{LE-ValidateSols.pgf}}
		\caption{Validation of numerically calculated Lane-Emden results}
	\end{figure}
\end{frame}

% \begin{frame}
% 	\frametitle{Comparing Partial TOV and LE results}
% 	\begin{itemize}
% 		\item Next, reduce terms present in TOV equation
% 		\begin{equation}
% 			\frac{\partial p}{\partial r} = -\underbrace{\underbrace{\underbrace{\underbrace{\frac{Gm\rho}{r^2}}_{i=0}\hspace{0.1cm}\left(1+\frac{p}{\rho c^2}\right)}_{i=1}\hspace{0.1cm}\left(\frac{4\pi r^3p}{mc^2}+1\right)}_{i=2}\hspace{0.1cm}\left(1-\frac{2Gm}{rc^2}\right)^{-1}}_{i=3}
% 		\end{equation}
% 		\item We expect $i=0$ to be the LE equation as derived earlier
% 	\end{itemize}
% \end{frame}
% 
% 
% \begin{frame}
% 	\begin{figure}
% 		\centering
% 		\scalebox{0.6}{\import{Part1-Main/Pictures/}{TOV-Terms.pgf}}
% 	\end{figure}
% \end{frame}


\subsection{Zero Values of TOV and LE equation}
\begin{frame}
	\frametitle{\insertsubsection}
% 	Show all the nice plots with the $n$ vs $\xi_0$ for different parameters plotted
	\begin{figure}
		\centering
		\scalebox{0.65}{\import{Part1-Main/Pictures/}{TOV-Exponents-LESubs-InitialVals-Database-PlotResults.pgf}}
		\caption{TOV and LE results for varying $A$ parameter of $\rho=Ap^{n/(n+1)}$}
	\end{figure}
\end{frame}

\begin{frame}
	\begin{figure}
		\centering
		\scalebox{0.65}{\import{Part1-Main/Pictures/}{TOV-Exponents-LESubs-InitialVals-Database-PlotResults-1.pgf}}
		\caption{TOV and LE solutions for varying parameter $p_0$.}
	\end{figure}
	Intersection: $r=\beta\xi$ with $4\pi\beta A^{n/(n+1)}=(n+1)\lambda^{1-1/n}$ is independent of $\lambda=\rho(p_0)$.
\end{frame}

\subsection{TOV Hypothesis}
\begin{frame}
	\frametitle{\insertsubsection}
	\begin{hypothesis}
		Given the TOV differential equation with $\rho=Ap^{\frac{n}{n+1}}$ and $p_0,A>0$
		\begin{alignat*}{3}
			\frac{\partial m}{\partial r} &= &&4\pi\rho r^2\\
			\frac{\partial p}{\partial r} &= -&&\frac{m\rho}{r^2}\left(1+\frac{p}{\rho}\right)\left(\frac{4\pi r^3p}{m}+1\right)\left(1-\frac{2m}{r}\right)^{-1}
		\end{alignat*}
		\setbeamertemplate{enumerate items}[default]
		\begin{enumerate}[<+->][(i)]
			\item There exists a solution for some $n_0\geq0$
			\item All solutions with same parameters $A,p_0$ and smaller exponent $n<n_0$ have a $p(r_0)$ for some $r_0>0$.
		\end{enumerate}
% 		To each combination $p_0,A>0$ there exists a $n_0\geq0$ such that all solutions of the TOV differential equation
% 		\begin{alignat*}{3}
% 			\frac{\partial m}{\partial r} &= &&4\pi\rho r^2\\
% 			\frac{\partial p}{\partial r} &= -&&\frac{m\rho}{r^2}\left(1+\frac{p}{\rho}\right)\left(\frac{4\pi r^3p}{m}+1\right)\left(1-\frac{2m}{r}\right)^{-1}
% 		\end{alignat*}
% 		with $\rho=Ap^{\frac{n}{n+1}}$ and a smaller $n\leq n_0$ has $p(\xi_0)=0$ for some $\xi_0\in\R_{>0}$.
	\end{hypothesis}
\end{frame}


% \subsection{Summary of Numerical Results}
% \begin{frame}
% 	\frametitle{\insertsubsection}
% 	\begin{itemize}
% 		\item Numerical and Exact results agree well where exact solutions are known
% 		
% 	\end{itemize}
% \end{frame}

\section{Exact Results}

% \subsection{Existance LE}
% \begin{frame}
% 	1 Frame\\
% 	Use Paper to show existance
% \end{frame}

\subsection{New Exact Solution for n=2}
\begin{frame}
	\frametitle{\insertsubsection}
	\begin{itemize}[<+->]
		\item Found new solution for $n=2$ by using simple power-series $\theta=\sum a_m\xi^m$
		\begin{equation}
			\frac{1}{\xi^2}\frac{\partial}{\partial\xi}\left(\xi^2\frac{\partial\theta}{\partial\xi}\right) +\theta^2 =0
		\end{equation}
		\item Apply Cauchy-Product formula and combine
		\begin{equation}
			(m+1)\sum\limits_{m=0}^\infty\left((m+2)a_{m+2}\xi^{m}+2a_{m+1}\xi^{m-1} + \sum\limits_{k=0}^m a_{m-k}a_k\xi^m\right) = 0
		\end{equation}
		\begin{theorem}
			The odd coefficients $a_{2m+1}$ vanish.
		\end{theorem}
		\item Proof by induction. Start at $\left.\partial\theta\right|_{\xi=0}=a_1=0$ is clear.
	\end{itemize}
\end{frame}


\begin{frame}
	\begin{itemize}[<+->]
		\item Obtain recursive Formula for coefficients
		\begin{equation}
			a_{2m+2}=b_{m+1}= -\frac{1}{(m+2)(m+3)}\sum\limits_{k=0}^mb_{m-k}b_k
		\end{equation}
		\begin{theorem}
			The series $\theta=\sum\limits_{m=0}^\infty b_{m}\xi^{2m}$ converges for $\xi<1$. % TODO was heißt absolute Konvergenz auf Englisch?
		\end{theorem}
		\begin{proof}
			We know that $a_0=1$. Use L'Hospital's rule for $\xi\rightarrow0$
			\begin{equation}
				0=\frac{2}{\xi}\frac{\partial\theta}{\partial\xi} + \frac{\partial^2\theta}{\partial\xi^2}+\theta^2\longrightarrow3\left.\frac{\partial^2\theta}{\partial\xi^2}\right|_{\xi=0}+\left.\theta\right|_{\xi=0}=0
			\end{equation}
			Thus $b_1=\theta_0/6$. By induction, we get $|b_{m+1}|\leq\frac{m\theta_0^2}{(m+2)(m+3)}$. Now use quotient criterion.
			% TODO find correct naming for Quotientenkriterium
		\end{proof}
	\end{itemize}
\end{frame}

\subsection{Hypothesis}
\begin{frame}
	\frametitle{\insertsubsection}
	\begin{figure}
		\scalebox{0.65}{\import{Part1-Main/Pictures/}{LE-ExactN2.pgf}}
		\caption{Exact solution $\theta$ and $(b_m)^{-1/(2m+2)}$ against $m$}
	\end{figure}
	\begin{itemize}%[<+->]
		\begin{hypothesis}
			The radius of convergence $R$ of the above calculated series is exactly the value at which $\theta(R)=0$.
		\end{hypothesis}
		\item Numerics: Calculation not optimized, reaches floating point limit
	\end{itemize}
\end{frame}


\subsection{Limiting Case TOV}
\begin{frame}[allowframebreaks]
% 	1 Frame\\
% 	Use Assumption that there exists a cont. solution of the TOV equation
	\frametitle{\insertsubsection}
	\begin{itemize}[<+->]
		\item Suppose the TOV equation has a solution that continuously depends on its parameters for $r\in[0,l)$ where $l$ may be $\infty$.
		\begin{theorem}
			Let $p_A$ be a solution of the TOV equation with $\rho=Ap^{1/\gamma}$. Then 
			\begin{equation}
				\lim_{A\rightarrow0}p_A=\frac{p_0}{2\pi p_0r^2+1}
			\end{equation}
		\end{theorem}
		\begin{proof}
			With $\partial m/\partial r = 4\pi Ap^{1/\gamma}r^2$, define $v:=m/A$
			\begin{equation}
				\frac{\partial p}{\partial r} = -\frac{p^{1/\gamma}}{r^2}\left(A+p^{1-1/\gamma}\right)\left(4\pi r^3p+vA\right)\left(1-\frac{2vA}{r}\right)^{-1}
			\end{equation}
		\end{proof}
	\end{itemize}
\end{frame}

\begin{frame}
	\begin{itemize}[<+->]
		\begin{proof}
			Then for $A=0$, we have
			\begin{equation}
				\frac{\partial p}{\partial r} = -4\pi rp^2
			\end{equation}
			The solution to this differential equation is $p=\frac{p_0}{2\pi p_0r^2+1}$.
		\end{proof}
		\item Problem: Assumption needs that TOV equation is solvable
		\item No experience with 2D singular ODEs.
		\item Easy transformation tricks for 1D singular ODEs not working.
	\end{itemize}
\end{frame}


% \subsection{}

\section{Outlook}

\begin{frame}
	\frametitle{\insertsection}
	\begin{itemize}[<+->]
		\item Plenty of information on numerical side
		\item Current work is (hopefully) well documented
		\item Should be able to obtain exact solution for $n\in\mathbb{N}, n>1$ analogously to $n=2$ cases
		\item Solvability of LE equation is known and well researched eg. \cite{quittnerSuperlinearParabolicProblems2007a}
		\item Solvability of TOV equation is complicated and subject of current research \cite{martinsExistenceClassificationPseudoAsymptotic2019, boonsermSolutionGeneratingTheorems2007}
	\end{itemize}
% 	1 Frame\\
% 	List what can be done and where to start if someone wants to continue this work
\end{frame}

\section*{Sources}

\begin{frame}[allowframebreaks]
	\sloppy 
	\printbibliography
\end{frame}


\end{document}
